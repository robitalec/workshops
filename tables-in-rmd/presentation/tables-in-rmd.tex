\documentclass[]{article}
\usepackage{lmodern}
\usepackage{amssymb,amsmath}
\usepackage{ifxetex,ifluatex}
\usepackage{fixltx2e} % provides \textsubscript
\ifnum 0\ifxetex 1\fi\ifluatex 1\fi=0 % if pdftex
  \usepackage[T1]{fontenc}
  \usepackage[utf8]{inputenc}
\else % if luatex or xelatex
  \ifxetex
    \usepackage{mathspec}
  \else
    \usepackage{fontspec}
  \fi
  \defaultfontfeatures{Ligatures=TeX,Scale=MatchLowercase}
\fi
% use upquote if available, for straight quotes in verbatim environments
\IfFileExists{upquote.sty}{\usepackage{upquote}}{}
% use microtype if available
\IfFileExists{microtype.sty}{%
\usepackage{microtype}
\UseMicrotypeSet[protrusion]{basicmath} % disable protrusion for tt fonts
}{}
\usepackage[margin=1in]{geometry}
\usepackage{hyperref}
\hypersetup{unicode=true,
            pdftitle={Tables in R Markdown},
            pdfauthor={Alec Robitaille},
            pdfborder={0 0 0},
            breaklinks=true}
\urlstyle{same}  % don't use monospace font for urls
\usepackage{color}
\usepackage{fancyvrb}
\newcommand{\VerbBar}{|}
\newcommand{\VERB}{\Verb[commandchars=\\\{\}]}
\DefineVerbatimEnvironment{Highlighting}{Verbatim}{commandchars=\\\{\}}
% Add ',fontsize=\small' for more characters per line
\usepackage{framed}
\definecolor{shadecolor}{RGB}{248,248,248}
\newenvironment{Shaded}{\begin{snugshade}}{\end{snugshade}}
\newcommand{\AlertTok}[1]{\textcolor[rgb]{0.94,0.16,0.16}{#1}}
\newcommand{\AnnotationTok}[1]{\textcolor[rgb]{0.56,0.35,0.01}{\textbf{\textit{#1}}}}
\newcommand{\AttributeTok}[1]{\textcolor[rgb]{0.77,0.63,0.00}{#1}}
\newcommand{\BaseNTok}[1]{\textcolor[rgb]{0.00,0.00,0.81}{#1}}
\newcommand{\BuiltInTok}[1]{#1}
\newcommand{\CharTok}[1]{\textcolor[rgb]{0.31,0.60,0.02}{#1}}
\newcommand{\CommentTok}[1]{\textcolor[rgb]{0.56,0.35,0.01}{\textit{#1}}}
\newcommand{\CommentVarTok}[1]{\textcolor[rgb]{0.56,0.35,0.01}{\textbf{\textit{#1}}}}
\newcommand{\ConstantTok}[1]{\textcolor[rgb]{0.00,0.00,0.00}{#1}}
\newcommand{\ControlFlowTok}[1]{\textcolor[rgb]{0.13,0.29,0.53}{\textbf{#1}}}
\newcommand{\DataTypeTok}[1]{\textcolor[rgb]{0.13,0.29,0.53}{#1}}
\newcommand{\DecValTok}[1]{\textcolor[rgb]{0.00,0.00,0.81}{#1}}
\newcommand{\DocumentationTok}[1]{\textcolor[rgb]{0.56,0.35,0.01}{\textbf{\textit{#1}}}}
\newcommand{\ErrorTok}[1]{\textcolor[rgb]{0.64,0.00,0.00}{\textbf{#1}}}
\newcommand{\ExtensionTok}[1]{#1}
\newcommand{\FloatTok}[1]{\textcolor[rgb]{0.00,0.00,0.81}{#1}}
\newcommand{\FunctionTok}[1]{\textcolor[rgb]{0.00,0.00,0.00}{#1}}
\newcommand{\ImportTok}[1]{#1}
\newcommand{\InformationTok}[1]{\textcolor[rgb]{0.56,0.35,0.01}{\textbf{\textit{#1}}}}
\newcommand{\KeywordTok}[1]{\textcolor[rgb]{0.13,0.29,0.53}{\textbf{#1}}}
\newcommand{\NormalTok}[1]{#1}
\newcommand{\OperatorTok}[1]{\textcolor[rgb]{0.81,0.36,0.00}{\textbf{#1}}}
\newcommand{\OtherTok}[1]{\textcolor[rgb]{0.56,0.35,0.01}{#1}}
\newcommand{\PreprocessorTok}[1]{\textcolor[rgb]{0.56,0.35,0.01}{\textit{#1}}}
\newcommand{\RegionMarkerTok}[1]{#1}
\newcommand{\SpecialCharTok}[1]{\textcolor[rgb]{0.00,0.00,0.00}{#1}}
\newcommand{\SpecialStringTok}[1]{\textcolor[rgb]{0.31,0.60,0.02}{#1}}
\newcommand{\StringTok}[1]{\textcolor[rgb]{0.31,0.60,0.02}{#1}}
\newcommand{\VariableTok}[1]{\textcolor[rgb]{0.00,0.00,0.00}{#1}}
\newcommand{\VerbatimStringTok}[1]{\textcolor[rgb]{0.31,0.60,0.02}{#1}}
\newcommand{\WarningTok}[1]{\textcolor[rgb]{0.56,0.35,0.01}{\textbf{\textit{#1}}}}
\usepackage{graphicx,grffile}
\makeatletter
\def\maxwidth{\ifdim\Gin@nat@width>\linewidth\linewidth\else\Gin@nat@width\fi}
\def\maxheight{\ifdim\Gin@nat@height>\textheight\textheight\else\Gin@nat@height\fi}
\makeatother
% Scale images if necessary, so that they will not overflow the page
% margins by default, and it is still possible to overwrite the defaults
% using explicit options in \includegraphics[width, height, ...]{}
\setkeys{Gin}{width=\maxwidth,height=\maxheight,keepaspectratio}
\IfFileExists{parskip.sty}{%
\usepackage{parskip}
}{% else
\setlength{\parindent}{0pt}
\setlength{\parskip}{6pt plus 2pt minus 1pt}
}
\setlength{\emergencystretch}{3em}  % prevent overfull lines
\providecommand{\tightlist}{%
  \setlength{\itemsep}{0pt}\setlength{\parskip}{0pt}}
\setcounter{secnumdepth}{0}
% Redefines (sub)paragraphs to behave more like sections
\ifx\paragraph\undefined\else
\let\oldparagraph\paragraph
\renewcommand{\paragraph}[1]{\oldparagraph{#1}\mbox{}}
\fi
\ifx\subparagraph\undefined\else
\let\oldsubparagraph\subparagraph
\renewcommand{\subparagraph}[1]{\oldsubparagraph{#1}\mbox{}}
\fi

%%% Use protect on footnotes to avoid problems with footnotes in titles
\let\rmarkdownfootnote\footnote%
\def\footnote{\protect\rmarkdownfootnote}

%%% Change title format to be more compact
\usepackage{titling}

% Create subtitle command for use in maketitle
\providecommand{\subtitle}[1]{
  \posttitle{
    \begin{center}\large#1\end{center}
    }
}

\setlength{\droptitle}{-2em}

  \title{Tables in R Markdown}
    \pretitle{\vspace{\droptitle}\centering\huge}
  \posttitle{\par}
    \author{Alec Robitaille}
    \preauthor{\centering\large\emph}
  \postauthor{\par}
      \predate{\centering\large\emph}
  \postdate{\par}
    \date{October 21 2019 {[}updated: October 17 2019{]}}


\begin{document}
\maketitle

\hypertarget{r-markdown}
\end{figure}

\href{https://bookdown.org/yihui/bookdown/}{bookdown: Authoring Books
and Technical Documents with R Markdown}

\begin{figure}
\centering
\includegraphics{https://bookdown.org/yihui/bookdown/images/cover.jpg}
\caption{:scale 30\%}
\end{figure}

{]}

.pull-right{[}

R Markdown combines: *
\href{https://github.com/yihui/knitr}{\texttt{knitr}} - executes code
chunks and converts from Rmd to md * \href{https://pandoc.org/}{Pandoc}
- converts from md to any format (pdf, docx, html, \ldots)

Helper packages include: *
\href{https://github.com/Rapporter/pander}{\texttt{pander}} - converts R
objects to markdown *
\href{https://github.com/yihui/tinytex}{\texttt{tinytex}} - provides a
cross-platform, lightweight LaTeX distribution (for rendering PDFs) *
\href{https://github.com/haozhu233/kableExtra}{\texttt{kableExtra}} -
styles \texttt{knitr} tables *
\href{https://github.com/rstudio/bookdown}{\texttt{bookdown}} - extends
R Markdown

.footnote{[}{]} {]}

\hypertarget{markdown-basics}{%
\section{Markdown basics}\label{markdown-basics}}

.pull-left{[}

\begin{Shaded}
\begin{Highlighting}[]
\FunctionTok{\# Header one}






\FunctionTok{\#\# Header two}



\NormalTok{*Italics*}


\NormalTok{**Bold**}





\NormalTok{![](images/beets.jpg)}
\end{Highlighting}
\end{Shaded}

{]}

.pull-right{[} \# Header one \#\# Header two

\emph{Italics}

\textbf{Bold}

\includegraphics{https://cdn-prod.medicalnewstoday.com/content/images/articles/277/277432/beetroot-on-a-white-background.jpg}
{]}

.footnote{[}\href{https://github.com/rstudio/cheatsheets/raw/master/rmarkdown-2.0.pdf}{Cheatsheet}{]}

??? image local or URL

class: example

\hypertarget{example-minimum-working-example}{%
\section{Example: minimum working
example}\label{example-minimum-working-example}}

\begin{enumerate}
\def\labelenumi{\arabic{enumi}.}
\item
  Open \texttt{examples/example-mwe.Rmd}
\item
  Knit the document
\item
  Check out the results
\item
  Add headers in the text
\end{enumerate}

title: ``Some title'' author: ``Dwigt'' date: ``2018-01-01'' output:
html\_document ---

\begin{verbatim}

* See `?rmarkdown::pdf_document`, `?rmarkdown::word_document`, ... for details


---
class: clear

\end{verbatim}

\hypertarget{registered-s3-method-overwritten-by-printr}{%
\subsection{Registered S3 method overwritten by
`printr':}\label{registered-s3-method-overwritten-by-printr}}

\hypertarget{method-from}{%
\subsection{method from}\label{method-from}}

\hypertarget{knit_print.data.frame-rmarkdown}{%
\subsection{knit\_print.data.frame
rmarkdown}\label{knit_print.data.frame-rmarkdown}}

\begin{verbatim}
\end{verbatim}

\hypertarget{convert-to-a-pdflatex-document}{%
\subsection{Convert to a PDF/LaTeX
document}\label{convert-to-a-pdflatex-document}}

\hypertarget{section}{%
\subsection{}\label{section}}

\hypertarget{usage}{%
\subsection{Usage:}\label{usage}}

\hypertarget{section-1}{%
\subsection{}\label{section-1}}

\hypertarget{pdf_documenttoc-false-toc_depth-2-number_sections-false}{%
\subsection{pdf\_document(toc = FALSE, toc\_depth = 2, number\_sections
=
FALSE,}\label{pdf_documenttoc-false-toc_depth-2-number_sections-false}}

\hypertarget{fig_width-6.5-fig_height-4.5-fig_crop-true}{%
\subsection{fig\_width = 6.5, fig\_height = 4.5, fig\_crop =
TRUE,}\label{fig_width-6.5-fig_height-4.5-fig_crop-true}}

\hypertarget{fig_caption-true-dev-pdf-df_print-default}{%
\subsection{fig\_caption = TRUE, dev = ``pdf'', df\_print =
``default'',}\label{fig_caption-true-dev-pdf-df_print-default}}

\hypertarget{highlight-default-template-default-keep_tex-false}{%
\subsection{highlight = ``default'', template = ``default'', keep\_tex =
FALSE,}\label{highlight-default-template-default-keep_tex-false}}

\hypertarget{keep_md-false-latex_engine-pdflatex}{%
\subsection{keep\_md = FALSE, latex\_engine =
``pdflatex'',}\label{keep_md-false-latex_engine-pdflatex}}

\hypertarget{citation_package-cnone-natbib-biblatex-includes-null}{%
\subsection{citation\_package = c(``none'', ``natbib'', ``biblatex''),
includes =
NULL,}\label{citation_package-cnone-natbib-biblatex-includes-null}}

\hypertarget{md_extensions-null-output_extensions-null-pandoc_args-null}{%
\subsection{md\_extensions = NULL, output\_extensions = NULL,
pandoc\_args =
NULL,}\label{md_extensions-null-output_extensions-null-pandoc_args-null}}

\hypertarget{extra_dependencies-null}{%
\subsection{extra\_dependencies = NULL)}\label{extra_dependencies-null}}

\hypertarget{section-2}{%
\subsection{}\label{section-2}}

\hypertarget{latex_document}{%
\subsection{latex\_document(\ldots)}\label{latex_document}}

\hypertarget{section-3}{%
\subsection{}\label{section-3}}

\hypertarget{latex_fragment}{%
\subsection{latex\_fragment(\ldots)}\label{latex_fragment}}

\begin{verbatim}




---
class: example

# Example: YAML frontmatter 

```yaml
---
title: "Some title"
author: "Dwigt"
date: "2018-01-01"
output: html_document
---
\end{verbatim}

\begin{enumerate}
\def\labelenumi{\arabic{enumi}.}
\item
  Open \texttt{examples/example-mwe.Rmd}
\item
  Switch the output to \texttt{word\_document} or \texttt{pdf\_document}
\item
  Update the title and author fields
\item
  Add a table of contents
  (\href{https://bookdown.org/yihui/rmarkdown/pdf-document.html\#table-of-contents-1}{hint})
\item
  Knit the document!
\end{enumerate}

.footnote{[} See options relevant to each output type in the
\href{https://github.com/rstudio/cheatsheets/raw/master/rmarkdown-2.0.pdf}{cheatsheet}
or extended descriptions in the
\href{https://bookdown.org/yihui/rmarkdown}{R Markdown book}{]}

??? tabs not spaces

class: example

\hypertarget{example-code-chunks}{%
\section{Example: code chunks}\label{example-code-chunks}}

\begin{enumerate}
\def\labelenumi{\arabic{enumi}.}
\tightlist
\item
  Open \texttt{examples/example-mwe.Rmd}
\item
  Remove the base R plot
\item
  Load \texttt{ggplot2}
\item
  Plot the cars dataset with \texttt{qplot()} or \texttt{ggplot()} e.g.:
  \texttt{qplot(speed,\ dist,\ data\ =\ cars)}
\item
  Set the figure width and figure height
  (\href{https://yihui.name/knitr/options/\#plots}{hint})
\item
  Knit the document!
\end{enumerate}

\hypertarget{models}{%
\section{Models}\label{models}}

\begin{itemize}
\tightlist
\item
  \texttt{broom}
\item
  Ben Bolker's \texttt{broom.mixed}
\item
  \texttt{pander}
\end{itemize}

title: ``Some title'' author: ``Dwigt'' date: ``2018-01-01'' output:
word\_document: * reference\_docx: styles.docx ---

\begin{verbatim}



1. Set desired styles in the reference docx (`examples/styles.docx`)
1. Add path in YAML
1. Knit the document!




---
# Captions with text references

1. Switch the output format to `bookdown::word_document2`
1. Use `(ref:label)` syntax to set the caption's label
1. Pass the label to `fig.cap` chunk option


````markdown
A normal paragraph.

(ref:fig1) A scatterplot of the data `cars` using base R graphics. 

```{r, fig.cap='(ref:fig1)'}
plot(cars)
\end{verbatim}

Another paragraph.

(ref:tab1) A summary table of the data \texttt{cars} using base R
graphics.

\texttt{\{r,\ fig.cap=\textquotesingle{}(ref:tab1)\textquotesingle{}\}\ knitr::kable(summary(cars))}
````

??? The syntax for a text reference is (ref:label) text, where label is
a unique label throughout the document for text. It must be in a
separate paragraph with empty lines above and below it. The paragraph
must not be wrapped into multiple lines, and should not end with a white
space.

\hypertarget{advanced}{%
\section{Advanced}\label{advanced}}

\begin{itemize}
\item
  Using LaTeX include\_headers
\item
  including pictures in tables wicked cool rmd knitr kable stuff
\item
  using \includegraphics{} syntax and LaTeX \includegraphics{}
\item
  figure out a word document solution
  \url{https://stackoverflow.com/questions/25106481/add-an-image-to-a-table-like-output-in-r}
\end{itemize}

\hypertarget{resource}{%
\section{Resource}\label{resource}}

\href{https://rmarkdown.rstudio.com/gallery.html}{R Markdown Gallery}

\href{https://github.com/rstudio/cheatsheets/raw/master/rmarkdown-2.0.pdf}{R
Markdown Cheatsheet}

\href{https://slides.yihui.name/2017-rstudio-conf-ext-rmd-Yihui-Xie.html\#1}{Yihui
Xie's Customizing \& Extending R Markdown}

\href{https://github.com/yihui/knitr}{\texttt{knitr}}

\href{https://bookdown.org/yihui/rmarkdown/}{R Markdown: The Definitive
Guide}

\includegraphics{https://d33wubrfki0l68.cloudfront.net/15c2850da53a2f8d8a1a30aa003bda21155e16ec/db869/images/logo-tinytex.png}

\begin{figure}
\centering
\includegraphics{https://bookdown.org/yihui/rmarkdown/images/hex-rmarkdown.png}
\caption{:scale 20\%}
\end{figure}

\includegraphics{images/knit-logo.png}
\includegraphics{images/pandoc.png}
\includegraphics{https://bookdown.org/yihui/bookdown/images/logo.png}

\url{https://rmarkdown.rstudio.com/articles_docx.html}

\begin{center}\rule{0.5\linewidth}{\linethickness}\end{center}

\hypertarget{setup}{%
\section{Setup}\label{setup}}

\begin{enumerate}
\def\labelenumi{\arabic{enumi}.}
\tightlist
\item
  Install packages
\end{enumerate}

\begin{itemize}
\tightlist
\item
  knitr, tinytex, rmarkdown, bookdown, kableExtra, pander
\end{itemize}

\begin{enumerate}
\def\labelenumi{\arabic{enumi}.}
\setcounter{enumi}{1}
\tightlist
\item
  Install tinytex
\end{enumerate}

\begin{itemize}
\tightlist
\item
  \texttt{tinytex::install\_tinytex()}
\end{itemize}


\end{document}
